\documentclass[a4paper, spanish]{extarticle} % Clase de documento tipo artículo extendido, papel A4, idioma español
\usepackage[spanish]{babel} % Traduce elementos automáticos al español
\usepackage[T1]{fontenc} % Codificación de fuentes para caracteres acentuados
\usepackage{geometry}  % Permite modificar los márgenes de la página
\usepackage{tcolorbox} % Cajas de color personalizables (para secciones)
\usepackage{xcolor} % Definición y uso de colores
\usepackage{sourcesanspro} % Fuente sans-serif moderna
\usepackage{lmodern} % Fuente Latin Modern
\usepackage{microtype} % Mejora la tipografía (espaciado, guiones, etc.)
\usepackage{expl3} % Sintaxis y utilidades avanzadas de LaTeX3
\usepackage{xparse} % Definición avanzada de comandos
\usepackage{amssymb} % Símbolos matemáticos adicionales

% Modifica el tamaño y espaciado de la fuente normal del documento
\makeatletter
\renewcommand\normalsize{%
   \@setfontsize\normalsize{10}{12}%  % 10pt de tamaño, 12pt de interlineado
   \abovedisplayskip 10\p@ \@plus2\p@ \@minus5\p@
   \abovedisplayshortskip \z@ \@plus3\p@
   \belowdisplayshortskip 6\p@ \@plus3\p@ \@minus3\p@
   \belowdisplayskip \abovedisplayskip
   \let\@listi\@listI}
\normalsize
\makeatother

% --------- MACROS para títulos estilizados ---------
\ExplSyntaxOn
% Crea títulos grandes y medianos para letras y números (usado en el título principal)
\NewDocumentCommand{\medstyledtitle}{m}
{
  \group_begin:
  \str_map_inline:nn {#1}
  {
    \__medstyledtitle_process:n {##1}
  }
  \group_end:
}

\cs_new_protected:Nn \__medstyledtitle_process:n
{
  \str_case_e:nnF { \tl_upper_case:n {#1} }
  {
    % Definición de formato para letras, números y símbolos
    {A}{\textbf{\fontsize{24pt}{26pt}\selectfont A}}
    {B}{\textbf{\fontsize{24pt}{26pt}\selectfont B}}
    {C}{\textbf{\fontsize{24pt}{26pt}\selectfont C}}
    {D}{\textbf{\fontsize{24pt}{26pt}\selectfont D}}
    {E}{\textbf{\fontsize{24pt}{26pt}\selectfont E}}
    {F}{\textbf{\fontsize{24pt}{26pt}\selectfont F}}
    {G}{\textbf{\fontsize{24pt}{26pt}\selectfont G}}
    {H}{\textbf{\fontsize{24pt}{26pt}\selectfont H}}
    {I}{\textbf{\fontsize{24pt}{26pt}\selectfont I}}
    {J}{\textbf{\fontsize{24pt}{26pt}\selectfont J}}
    {K}{\textbf{\fontsize{24pt}{26pt}\selectfont K}}
    {L}{\textbf{\fontsize{24pt}{26pt}\selectfont L}}
    {M}{\textbf{\fontsize{24pt}{26pt}\selectfont M}}
    {N}{\textbf{\fontsize{24pt}{26pt}\selectfont N}}
    {O}{\textbf{\fontsize{24pt}{26pt}\selectfont O}}
    {P}{\textbf{\fontsize{24pt}{26pt}\selectfont P}}
    {Q}{\textbf{\fontsize{24pt}{26pt}\selectfont Q}}
    {R}{\textbf{\fontsize{24pt}{26pt}\selectfont R}}
    {S}{\textbf{\fontsize{24pt}{26pt}\selectfont S}}
    {T}{\textbf{\fontsize{24pt}{26pt}\selectfont T}}
    {U}{\textbf{\fontsize{24pt}{26pt}\selectfont U}}
    {V}{\textbf{\fontsize{24pt}{26pt}\selectfont V}}
    {W}{\textbf{\fontsize{24pt}{26pt}\selectfont W}}
    {X}{\textbf{\fontsize{24pt}{26pt}\selectfont X}}
    {Y}{\textbf{\fontsize{24pt}{26pt}\selectfont Y}}
    {Z}{\textbf{\fontsize{24pt}{26pt}\selectfont Z}}
    {Á}{\textbf{\fontsize{24pt}{26pt}\selectfont Á}}
    {É}{\textbf{\fontsize{24pt}{26pt}\selectfont É}}
    {Í}{\textbf{\fontsize{24pt}{26pt}\selectfont Í}}
    {Ó}{\textbf{\fontsize{24pt}{26pt}\selectfont Ó}}
    {Ú}{\textbf{\fontsize{24pt}{26pt}\selectfont Ú}}
    {Ñ}{\textbf{\fontsize{24pt}{26pt}\selectfont Ñ}}
    
    % Números (0-9)
    {0}{\textbf{\fontsize{24pt}{26pt}\selectfont 0}}
    {1}{\textbf{\fontsize{24pt}{26pt}\selectfont 1}}
    {2}{\textbf{\fontsize{24pt}{26pt}\selectfont 2}}
    {3}{\textbf{\fontsize{24pt}{26pt}\selectfont 3}}
    {4}{\textbf{\fontsize{24pt}{26pt}\selectfont 4}}
    {5}{\textbf{\fontsize{24pt}{26pt}\selectfont 5}}
    {6}{\textbf{\fontsize{24pt}{26pt}\selectfont 6}}
    {7}{\textbf{\fontsize{24pt}{26pt}\selectfont 7}}
    {8}{\textbf{\fontsize{24pt}{26pt}\selectfont 8}}
    {9}{\textbf{\fontsize{24pt}{26pt}\selectfont 9}}
    
    % Símbolos especiales
    { }{\hspace{0.45em}}
    {:}{\fontsize{17pt}{19pt}\selectfont :}
  }
  {\fontsize{17pt}{19pt}\selectfont #1} % Por defecto, tamaño mediano
}

\NewDocumentCommand{\smallmedstyledtitle}{m}
{
  \group_begin:
  \str_map_inline:nn {#1}
  {
    \__smallmedstyledtitle_process:n {##1}
  }
  \group_end:
}

\cs_new_protected:Nn \__smallmedstyledtitle_process:n
{
  \str_case_e:nnF { \tl_upper_case:n {#1} }
  {
    % Letras (mayúsculas y acentuadas)
    {A}{\textbf{\fontsize{19pt}{21pt}\selectfont A}}
    {B}{\textbf{\fontsize{19pt}{21pt}\selectfont B}}
    {C}{\textbf{\fontsize{19pt}{21pt}\selectfont C}}
    {D}{\textbf{\fontsize{19pt}{21pt}\selectfont D}}
    {E}{\textbf{\fontsize{19pt}{21pt}\selectfont E}}
    {F}{\textbf{\fontsize{19pt}{21pt}\selectfont F}}
    {G}{\textbf{\fontsize{19pt}{21pt}\selectfont G}}
    {H}{\textbf{\fontsize{19pt}{21pt}\selectfont H}}
    {I}{\textbf{\fontsize{19pt}{21pt}\selectfont I}}
    {J}{\textbf{\fontsize{19pt}{21pt}\selectfont J}}
    {K}{\textbf{\fontsize{19pt}{21pt}\selectfont K}}
    {L}{\textbf{\fontsize{19pt}{21pt}\selectfont L}}
    {M}{\textbf{\fontsize{19pt}{21pt}\selectfont M}}
    {N}{\textbf{\fontsize{19pt}{21pt}\selectfont N}}
    {O}{\textbf{\fontsize{19pt}{21pt}\selectfont O}}
    {P}{\textbf{\fontsize{19pt}{21pt}\selectfont P}}
    {Q}{\textbf{\fontsize{19pt}{21pt}\selectfont Q}}
    {R}{\textbf{\fontsize{19pt}{21pt}\selectfont R}}
    {S}{\textbf{\fontsize{19pt}{21pt}\selectfont S}}
    {T}{\textbf{\fontsize{19pt}{21pt}\selectfont T}}
    {U}{\textbf{\fontsize{19pt}{21pt}\selectfont U}}
    {V}{\textbf{\fontsize{19pt}{21pt}\selectfont V}}
    {W}{\textbf{\fontsize{19pt}{21pt}\selectfont W}}
    {X}{\textbf{\fontsize{19pt}{21pt}\selectfont X}}
    {Y}{\textbf{\fontsize{19pt}{21pt}\selectfont Y}}
    {Z}{\textbf{\fontsize{19pt}{21pt}\selectfont Z}}
    {Á}{\textbf{\fontsize{19pt}{21pt}\selectfont Á}}
    {É}{\textbf{\fontsize{19pt}{21pt}\selectfont É}}
    {Í}{\textbf{\fontsize{19pt}{21pt}\selectfont Í}}
    {Ó}{\textbf{\fontsize{19pt}{21pt}\selectfont Ó}}
    {Ú}{\textbf{\fontsize{19pt}{21pt}\selectfont Ú}}
    {Ñ}{\textbf{\fontsize{19pt}{21pt}\selectfont Ñ}}
    
    % Números (0-9)
    {0}{\textbf{\fontsize{19pt}{21pt}\selectfont 0}}
    {1}{\textbf{\fontsize{19pt}{21pt}\selectfont 1}}
    {2}{\textbf{\fontsize{19pt}{21pt}\selectfont 2}}
    {3}{\textbf{\fontsize{19pt}{21pt}\selectfont 3}}
    {4}{\textbf{\fontsize{19pt}{21pt}\selectfont 4}}
    {5}{\textbf{\fontsize{19pt}{21pt}\selectfont 5}}
    {6}{\textbf{\fontsize{19pt}{21pt}\selectfont 6}}
    {7}{\textbf{\fontsize{19pt}{21pt}\selectfont 7}}
    {8}{\textbf{\fontsize{19pt}{21pt}\selectfont 8}}
    {9}{\textbf{\fontsize{19pt}{21pt}\selectfont 9}}
    
    % Símbolos especiales
    { }{\hspace{0.45em}}
    {:}{\fontsize{17pt}{19pt}\selectfont :}
  }
  {\fontsize{17pt}{19pt}\selectfont #1}
}
\ExplSyntaxOff

% --------- Configuración tipográfica general ---------
\renewcommand{\familydefault}{\sfdefault} % Fuente sans-serif por defecto
\AtBeginDocument{%
  \SetMathAlphabet{\mathrm}{normal}{T1}{lmr}{m}{n}   % Configuración de fuentes matemáticas
  \SetMathAlphabet{\mathbf}{normal}{T1}{lmr}{b}{n}
  \SetMathAlphabet{\mathit}{normal}{T1}{lmr}{m}{it}
  \SetMathAlphabet{\mathtt}{normal}{T1}{lmtt}{m}{n}
}

\geometry{margin=2cm} % Márgenes de 2cm en toda la página

% Definición de colores personalizados para las cajas de secciones
\definecolor{color1}{HTML}{4E79A7}
\definecolor{color2}{HTML}{F28E2B}
\definecolor{color3}{HTML}{E15759}
\definecolor{color4}{HTML}{59A14F}
\definecolor{color5}{HTML}{EDC948}
\definecolor{color6}{HTML}{F28E2B}

% Configuración de la caja de sección personalizada usando tcolorbox
\tcbuselibrary{skins, breakable}
\newtcolorbox{seccionbox}[2]{
  colback=#1!5!white, % Color de fondo suave
  colframe=#1!90!black, % Borde oscuro del mismo color
  fonttitle=\bfseries\large, % Fuente del título en negrita y grande
  title={\sffamily #2}, % Título en sans-serif
  breakable,
  enhanced,
  boxrule=1.5pt, % Grosor del borde
  arc=4pt, % Esquinas redondeadas
  attach boxed title to top left={xshift=5mm, yshift=-2mm},
  boxed title style={
    colback=#1!80!black,
    arc=3pt,
    outer arc=3pt
  }
}

% --------- Contadores y comandos para numeración automática de secciones ---------
\newcounter{mainsec}
\newcounter{subsec}[mainsec]
\newcounter{subsubsec}[subsec]
\newcounter{subsubsubsec}[subsubsec]

\newcommand{\mainnum}{\stepcounter{mainsec}\arabic{mainsec}}
\newcommand{\subnum}{\stepcounter{subsec}\arabic{mainsec}.\arabic{subsec}}
\newcommand{\subsubnum}{\stepcounter{subsubsec}\arabic{mainsec}.\arabic{subsec}.\arabic{subsubsec}}
\newcommand{\subsubsubnum}{\stepcounter{subsubsubsec}\arabic{mainsec}.\arabic{subsec}.\arabic{subsubsec}.\arabic{subsubsubsec}}
\newcommand{\numstyle}[1]{\textbf{\large #1}\quad} % Estilo destacado para la numeración

% --------- Título personalizado con letras grandes ---------
\title{\vspace{-2em}%
  \medstyledtitle{F}%
  \smallmedstyledtitle{ÍSICA}%
  \hspace{0.45em}%
  \medstyledtitle{N}%
  \smallmedstyledtitle{EWTONIANA}%
}
\author{}
\date{}

\begin{document}
\maketitle

% Reinicio de contadores
\setcounter{mainsec}{0}
\setcounter{subsec}{0}
\setcounter{subsubsec}{0}
\setcounter{subsubsubsec}{0}

% ========= SECCIÓN 1: MAGNITUDES FÍSICAS Y UNIDADES =========
\begin{seccionbox}{color1}{\mainnum. Magnitudes Físicas y Unidades}
\vspace{0.5em}
\numstyle{\subnum} Magnitudes adimensionales, fundamentales y derivadas.\par
\numstyle{\subnum} Sistemas de unidades:\par
\numstyle{\subsubnum} Sistema Internacional (SI).\par
\numstyle{\subsubnum} Sistema Inglés.\par
\numstyle{\subnum} Prefijos métricos en base 10.\par
\numstyle{\subnum} Conversión de unidades y notación científica.\par
\numstyle{\subnum} Incertidumbre y cifras significativas.\par
\numstyle{\subnum} Estimaciones, órdenes de magnitud y regresión lineal.\par
\numstyle{\subnum} Análisis dimensional.\par  
\end{seccionbox}

\setcounter{subsec}{0}
\setcounter{subsubsec}{0}
\setcounter{subsubsubsec}{0}

% ========= SECCIÓN 2: VECTORES =========
\begin{seccionbox}{color2}{\mainnum. Vectores}
\vspace{0.5em}
\numstyle{\subnum} Vectores y escalares.\par
\numstyle{\subnum} Vectores en una, dos y tres dimensiones.\par
\numstyle{\subnum} Propiedades y operaciones básicas:\par
\numstyle{\subsubnum} Magnitud.\par
\numstyle{\subsubnum} Adición y sustracción vectorial.\par
\numstyle{\subsubnum} Multiplicación por escalar.\par
\numstyle{\subnum} Productos vectoriales:\par
\numstyle{\subsubnum} Producto punto ($\vec{a} \cdot \vec{b}$ -- Escalar).\par
\numstyle{\subsubnum} Producto cruz ($\vec{a} \times \vec{b}$ -- Vectorial).\par
\numstyle{\subnum} Sistemas de coordenadas:\par
\numstyle{\subsubnum} Cartesianas.\par
\numstyle{\subsubnum} Polares.\par
\numstyle{\subsubnum} Conversión entre sistemas.\par
\end{seccionbox}

\setcounter{subsec}{0}
\setcounter{subsubsec}{0}
\setcounter{subsubsubsec}{0}

% ========= SECCIÓN 3: CINEMÁTICA =========
\begin{seccionbox}{color3}{\mainnum. Cinemática}
\vspace{0.5em}  
\numstyle{\subnum} Movimiento Unidimensional:\par
\numstyle{\subsubnum} Conceptos básicos:\par
\numstyle{\subsubsubnum} Posición y desplazamiento.\par
\numstyle{\subsubsubnum} Velocidad y aceleración promedio e instantánea.\par
\numstyle{\subsubnum} Movimiento Rectilíneo Uniforme (MRU).\par
\numstyle{\subsubnum} Movimiento Rectilíneo Uniformemente Acelerado (MRUA).\par
\numstyle{\subsubnum} Caída libre como caso de MRUA.\par
\numstyle{\subnum} Movimiento Bidimensional:\par
\numstyle{\subsubnum} Conceptos elementales:\par
\numstyle{\subsubsubnum} Vectores de posición, velocidad y aceleración.\par
\numstyle{\subsubsubnum} Componentes independientes del movimiento en dos dimensiones.\par
\numstyle{\subsubsubnum} Alcance máximo y altura máxima.\par
\numstyle{\subsubnum} Movimiento semiparabólico.\par
\numstyle{\subsubnum} Movimiento parabólico completo.\par
\numstyle{\subnum} Movimiento Circular:\par
\numstyle{\subsubnum} Conceptos angulares:\par
\numstyle{\subsubsubnum} Periodo y frecuencia.\par
\numstyle{\subsubsubnum} Desplazamiento, velocidad y aceleración angular.\par
\numstyle{\subsubsubnum} Relación entre variables lineales y angulares.\par
\numstyle{\subsubsubnum} Aceleración centrípeta.\par
\numstyle{\subsubnum} Movimiento Circular Uniforme (MCU).\par
\numstyle{\subsubnum} Movimiento circular no uniforme.\par
\end{seccionbox}

\setcounter{subsec}{0}
\setcounter{subsubsec}{0}
\setcounter{subsubsubsec}{0}

% ========= SECCIÓN 4: DINÁMICA =========
\begin{seccionbox}{color4}{\mainnum. Dinámica}
\vspace{0.5em}
\numstyle{\subnum} Primera Ley de Newton: Ley de Inercia.\par
\numstyle{\subnum} Segunda Ley de Newton:\par
\numstyle{\subsubnum} Estática o Equilibrio: $\sum\vec{F} = \vec{0}$.\par
\numstyle{\subsubnum} Dinámica: $\sum\vec{F} = m\vec{a}$.\par
\numstyle{\subnum} Tercera Ley de Newton: Acción y Reacción.\par
\numstyle{\subnum} Aplicaciones:\par
\numstyle{\subsubnum} Diagramas de Cuerpo Libre (DCL).\par
\numstyle{\subsubnum} Tipos de fuerzas:\par
\numstyle{\subsubsubnum} Peso (fuerza gravitatoria) y normal.\par
\numstyle{\subsubsubnum} Tensión.\par
\numstyle{\subsubsubnum} Fricción (estática y cinética).\par
\numstyle{\subsubsubnum} Fuerza elástica (Ley de Hooke).\par
\numstyle{\subsubnum} Sistemas en equilibrio estático.\par
\numstyle{\subsubnum} Sistemas en movimiento dinámico (planos inclinados, poleas).\par
\numstyle{\subnum} Dinámica Rotacional:\par
\numstyle{\subsubnum} Torque (momento de fuerza).\par
\numstyle{\subsubnum} Momento de inercia.\par
\numstyle{\subsubnum} Segunda Ley para rotación: $\sum\tau = I\alpha$.\par
\numstyle{\subsubnum} Energía cinética rotacional.\par
\numstyle{\subsubnum} Equilibrio rotacional.\par
\numstyle{\subsubnum} Aplicaciones:\par
\numstyle{\subsubsubnum} Palancas (equilibrio y movimiento).\par
\numstyle{\subsubsubnum} Poleas con momento de inercia.\par
\end{seccionbox}

\setcounter{subsec}{0}
\setcounter{subsubsec}{0}
\setcounter{subsubsubsec}{0}

% ========= SECCIÓN 5: TRABAJO Y ENERGÍA =========
\begin{seccionbox}{color5}{\mainnum. Trabajo y Energía}
\vspace{0.5em}
\numstyle{\subnum} Concepto de trabajo mecánico.\par
\numstyle{\subnum} Trabajo por una fuerza constante y variable.\par
\numstyle{\subnum} Energía mecánica:\par
\numstyle{\subsubnum} Energía cinética (traslacional y rotacional).\par
\numstyle{\subsubnum} Energía potencial (gravitatoria y elástica).\par
\numstyle{\subnum} Teorema trabajo-energía.\par
\numstyle{\subnum} Principio de conservación de la energía mecánica.\par
\numstyle{\subsubnum} Ejemplos y aplicaciones:\par
\numstyle{\subsubsubnum} Caída libre.\par
\numstyle{\subsubsubnum} Planos inclinados.\par
\numstyle{\subsubsubnum} Péndulos.\par
\numstyle{\subnum} Fuerzas conservativas y no conservativas.\par
\numstyle{\subnum} Potencia mecánica.\par
\end{seccionbox}

\setcounter{subsec}{0}
\setcounter{subsubsec}{0}
\setcounter{subsubsubsec}{0}

% ========= SECCIÓN 6: SISTEMAS DE PARTÍCULAS =========
\begin{seccionbox}{color6}{\mainnum. Sistemas de Partículas y Cantidad de Movimiento}
\vspace{0.5em}
\numstyle{\subnum} Centro de masa.\par
\numstyle{\subnum} Momento lineal (cantidad de movimiento).\par
\numstyle{\subnum} Impulso mecánico.\par
\numstyle{\subnum} Teorema impulso-momento.\par
\numstyle{\subnum} Conservación del momento lineal.\par
\numstyle{\subnum} Colisiones o choques:\par
\numstyle{\subsubnum} Elásticas.\par
\numstyle{\subsubnum} Inelásticas.\par
\numstyle{\subsubnum} Perfectamente inelásticas.\par
\numstyle{\subsubnum} Colisiones en dos dimensiones.\par
\end{seccionbox}

\end{document}
