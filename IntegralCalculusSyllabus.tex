\documentclass[a4paper, spanish]{extarticle} % Clase de documento tipo artículo extendido, papel A4, idioma español
\usepackage[spanish]{babel} % Traduce elementos automáticos al español
\usepackage[T1]{fontenc} % Codificación de fuentes para caracteres acentuados
\usepackage{geometry}  % Permite modificar los márgenes de la página
\usepackage{tcolorbox} % Cajas de color personalizables (para secciones)
\usepackage{xcolor} % Definición y uso de colores
\usepackage{sourcesanspro} % Fuente sans-serif moderna
\usepackage{lmodern} % Fuente Latin Modern
\usepackage{microtype} % Mejora la tipografía (espaciado, guiones, etc.)
\usepackage{expl3} % Sintaxis y utilidades avanzadas de LaTeX3
\usepackage{xparse} % Definición avanzada de comandos
\usepackage{amssymb} % Símbolos matemáticos adicionales

% Modifica el tamaño y espaciado de la fuente normal del documento
\makeatletter
\renewcommand\normalsize{%
   \@setfontsize\normalsize{10}{12}%  % 10pt de tamaño, 12pt de interlineado
   \abovedisplayskip 10\p@ \@plus2\p@ \@minus5\p@
   \abovedisplayshortskip \z@ \@plus3\p@
   \belowdisplayshortskip 6\p@ \@plus3\p@ \@minus3\p@
   \belowdisplayskip \abovedisplayskip
   \let\@listi\@listI}
\normalsize
\makeatother

% --------- MACROS para títulos estilizados ---------
\ExplSyntaxOn
% Crea títulos grandes y medianos para letras y números (usado en el título principal)
\NewDocumentCommand{\medstyledtitle}{m}
{
  \group_begin:
  \str_map_inline:nn {#1}
  {
    \__medstyledtitle_process:n {##1}
  }
  \group_end:
}

\cs_new_protected:Nn \__medstyledtitle_process:n
{
  \str_case_e:nnF { \tl_upper_case:n {#1} }
  {
    % Definición de formato para letras, números y símbolos
    {A}{\textbf{\fontsize{24pt}{26pt}\selectfont A}}
    {B}{\textbf{\fontsize{24pt}{26pt}\selectfont B}}
    {C}{\textbf{\fontsize{24pt}{26pt}\selectfont C}}
    {D}{\textbf{\fontsize{24pt}{26pt}\selectfont D}}
    {E}{\textbf{\fontsize{24pt}{26pt}\selectfont E}}
    {F}{\textbf{\fontsize{24pt}{26pt}\selectfont F}}
    {G}{\textbf{\fontsize{24pt}{26pt}\selectfont G}}
    {H}{\textbf{\fontsize{24pt}{26pt}\selectfont H}}
    {I}{\textbf{\fontsize{24pt}{26pt}\selectfont I}}
    {J}{\textbf{\fontsize{24pt}{26pt}\selectfont J}}
    {K}{\textbf{\fontsize{24pt}{26pt}\selectfont K}}
    {L}{\textbf{\fontsize{24pt}{26pt}\selectfont L}}
    {M}{\textbf{\fontsize{24pt}{26pt}\selectfont M}}
    {N}{\textbf{\fontsize{24pt}{26pt}\selectfont N}}
    {O}{\textbf{\fontsize{24pt}{26pt}\selectfont O}}
    {P}{\textbf{\fontsize{24pt}{26pt}\selectfont P}}
    {Q}{\textbf{\fontsize{24pt}{26pt}\selectfont Q}}
    {R}{\textbf{\fontsize{24pt}{26pt}\selectfont R}}
    {S}{\textbf{\fontsize{24pt}{26pt}\selectfont S}}
    {T}{\textbf{\fontsize{24pt}{26pt}\selectfont T}}
    {U}{\textbf{\fontsize{24pt}{26pt}\selectfont U}}
    {V}{\textbf{\fontsize{24pt}{26pt}\selectfont V}}
    {W}{\textbf{\fontsize{24pt}{26pt}\selectfont W}}
    {X}{\textbf{\fontsize{24pt}{26pt}\selectfont X}}
    {Y}{\textbf{\fontsize{24pt}{26pt}\selectfont Y}}
    {Z}{\textbf{\fontsize{24pt}{26pt}\selectfont Z}}
    {Á}{\textbf{\fontsize{24pt}{26pt}\selectfont Á}}
    {É}{\textbf{\fontsize{24pt}{26pt}\selectfont É}}
    {Í}{\textbf{\fontsize{24pt}{26pt}\selectfont Í}}
    {Ó}{\textbf{\fontsize{24pt}{26pt}\selectfont Ó}}
    {Ú}{\textbf{\fontsize{24pt}{26pt}\selectfont Ú}}
    {Ñ}{\textbf{\fontsize{24pt}{26pt}\selectfont Ñ}}
    
    % Números (0-9)
    {0}{\textbf{\fontsize{24pt}{26pt}\selectfont 0}}
    {1}{\textbf{\fontsize{24pt}{26pt}\selectfont 1}}
    {2}{\textbf{\fontsize{24pt}{26pt}\selectfont 2}}
    {3}{\textbf{\fontsize{24pt}{26pt}\selectfont 3}}
    {4}{\textbf{\fontsize{24pt}{26pt}\selectfont 4}}
    {5}{\textbf{\fontsize{24pt}{26pt}\selectfont 5}}
    {6}{\textbf{\fontsize{24pt}{26pt}\selectfont 6}}
    {7}{\textbf{\fontsize{24pt}{26pt}\selectfont 7}}
    {8}{\textbf{\fontsize{24pt}{26pt}\selectfont 8}}
    {9}{\textbf{\fontsize{24pt}{26pt}\selectfont 9}}
    
    % Símbolos especiales
    { }{\hspace{0.45em}}
    {:}{\fontsize{17pt}{19pt}\selectfont :}
  }
  {\fontsize{17pt}{19pt}\selectfont #1} % Por defecto, tamaño mediano
}

\NewDocumentCommand{\smallmedstyledtitle}{m}
{
  \group_begin:
  \str_map_inline:nn {#1}
  {
    \__smallmedstyledtitle_process:n {##1}
  }
  \group_end:
}

\cs_new_protected:Nn \__smallmedstyledtitle_process:n
{
  \str_case_e:nnF { \tl_upper_case:n {#1} }
  {
    % Letras (mayúsculas y acentuadas)
    {A}{\textbf{\fontsize{19pt}{21pt}\selectfont A}}
    {B}{\textbf{\fontsize{19pt}{21pt}\selectfont B}}
    {C}{\textbf{\fontsize{19pt}{21pt}\selectfont C}}
    {D}{\textbf{\fontsize{19pt}{21pt}\selectfont D}}
    {E}{\textbf{\fontsize{19pt}{21pt}\selectfont E}}
    {F}{\textbf{\fontsize{19pt}{21pt}\selectfont F}}
    {G}{\textbf{\fontsize{19pt}{21pt}\selectfont G}}
    {H}{\textbf{\fontsize{19pt}{21pt}\selectfont H}}
    {I}{\textbf{\fontsize{19pt}{21pt}\selectfont I}}
    {J}{\textbf{\fontsize{19pt}{21pt}\selectfont J}}
    {K}{\textbf{\fontsize{19pt}{21pt}\selectfont K}}
    {L}{\textbf{\fontsize{19pt}{21pt}\selectfont L}}
    {M}{\textbf{\fontsize{19pt}{21pt}\selectfont M}}
    {N}{\textbf{\fontsize{19pt}{21pt}\selectfont N}}
    {O}{\textbf{\fontsize{19pt}{21pt}\selectfont O}}
    {P}{\textbf{\fontsize{19pt}{21pt}\selectfont P}}
    {Q}{\textbf{\fontsize{19pt}{21pt}\selectfont Q}}
    {R}{\textbf{\fontsize{19pt}{21pt}\selectfont R}}
    {S}{\textbf{\fontsize{19pt}{21pt}\selectfont S}}
    {T}{\textbf{\fontsize{19pt}{21pt}\selectfont T}}
    {U}{\textbf{\fontsize{19pt}{21pt}\selectfont U}}
    {V}{\textbf{\fontsize{19pt}{21pt}\selectfont V}}
    {W}{\textbf{\fontsize{19pt}{21pt}\selectfont W}}
    {X}{\textbf{\fontsize{19pt}{21pt}\selectfont X}}
    {Y}{\textbf{\fontsize{19pt}{21pt}\selectfont Y}}
    {Z}{\textbf{\fontsize{19pt}{21pt}\selectfont Z}}
    {Á}{\textbf{\fontsize{19pt}{21pt}\selectfont Á}}
    {É}{\textbf{\fontsize{19pt}{21pt}\selectfont É}}
    {Í}{\textbf{\fontsize{19pt}{21pt}\selectfont Í}}
    {Ó}{\textbf{\fontsize{19pt}{21pt}\selectfont Ó}}
    {Ú}{\textbf{\fontsize{19pt}{21pt}\selectfont Ú}}
    {Ñ}{\textbf{\fontsize{19pt}{21pt}\selectfont Ñ}}
    
    % Números (0-9)
    {0}{\textbf{\fontsize{19pt}{21pt}\selectfont 0}}
    {1}{\textbf{\fontsize{19pt}{21pt}\selectfont 1}}
    {2}{\textbf{\fontsize{19pt}{21pt}\selectfont 2}}
    {3}{\textbf{\fontsize{19pt}{21pt}\selectfont 3}}
    {4}{\textbf{\fontsize{19pt}{21pt}\selectfont 4}}
    {5}{\textbf{\fontsize{19pt}{21pt}\selectfont 5}}
    {6}{\textbf{\fontsize{19pt}{21pt}\selectfont 6}}
    {7}{\textbf{\fontsize{19pt}{21pt}\selectfont 7}}
    {8}{\textbf{\fontsize{19pt}{21pt}\selectfont 8}}
    {9}{\textbf{\fontsize{19pt}{21pt}\selectfont 9}}
    
    % Símbolos especiales
    { }{\hspace{0.45em}}
    {:}{\fontsize{17pt}{19pt}\selectfont :}
  }
  {\fontsize{17pt}{19pt}\selectfont #1}
}
\ExplSyntaxOff

% --------- Configuración tipográfica general ---------
\renewcommand{\familydefault}{\sfdefault} % Fuente sans-serif por defecto
\AtBeginDocument{%
  \SetMathAlphabet{\mathrm}{normal}{T1}{lmr}{m}{n}   % Configuración de fuentes matemáticas
  \SetMathAlphabet{\mathbf}{normal}{T1}{lmr}{b}{n}
  \SetMathAlphabet{\mathit}{normal}{T1}{lmr}{m}{it}
  \SetMathAlphabet{\mathtt}{normal}{T1}{lmtt}{m}{n}
}

\geometry{margin=2cm} % Márgenes de 2cm en toda la página

% Definición de colores personalizados para las cajas de secciones
\definecolor{color1}{HTML}{4E79A7}
\definecolor{color2}{HTML}{F28E2B}
\definecolor{color3}{HTML}{E15759}
\definecolor{color4}{HTML}{59A14F}
\definecolor{color5}{HTML}{EDC948}
\definecolor{color6}{HTML}{F28E2B}

% Configuración de la caja de sección personalizada usando tcolorbox
\tcbuselibrary{skins, breakable}
\newtcolorbox{seccionbox}[2]{
  colback=#1!5!white, % Color de fondo suave
  colframe=#1!90!black, % Borde oscuro del mismo color
  fonttitle=\bfseries\large, % Fuente del título en negrita y grande
  title={\sffamily #2}, % Título en sans-serif
  breakable,
  enhanced,
  boxrule=1.5pt, % Grosor del borde
  arc=4pt, % Esquinas redondeadas
  attach boxed title to top left={xshift=5mm, yshift=-2mm},
  boxed title style={
    colback=#1!80!black,
    arc=3pt,
    outer arc=3pt
  }
}

% --------- Contadores y comandos para numeración automática de secciones ---------
\newcounter{mainsec}
\newcounter{subsec}[mainsec]
\newcounter{subsubsec}[subsec]
\newcounter{subsubsubsec}[subsubsec]

\newcommand{\mainnum}{\stepcounter{mainsec}\arabic{mainsec}}
\newcommand{\subnum}{\stepcounter{subsec}\arabic{mainsec}.\arabic{subsec}}
\newcommand{\subsubnum}{\stepcounter{subsubsec}\arabic{mainsec}.\arabic{subsec}.\arabic{subsubsec}}
\newcommand{\subsubsubnum}{\stepcounter{subsubsubsec}\arabic{mainsec}.\arabic{subsec}.\arabic{subsubsec}.\arabic{subsubsubsec}}
\newcommand{\numstyle}[1]{\textbf{\large #1}\quad} % Estilo destacado para la numeración

% --------- Título personalizado con letras grandes ---------
\title{\vspace{-2em}%
  \medstyledtitle{C}%
  \smallmedstyledtitle{ÁLCULO}%
  \hspace{0.45em}%
  \medstyledtitle{I}%
  \smallmedstyledtitle{NTEGRAL}%
}
\author{}
\date{}

\begin{document}
\maketitle

% Reinicio de contadores
\setcounter{mainsec}{0}
\setcounter{subsec}{0}
\setcounter{subsubsec}{0}
\setcounter{subsubsubsec}{0}

% ================= SECCIÓN 1 =================
\begin{seccionbox}{color1}{\mainnum. Introducción a la Integración}
\vspace{0.5em}
\numstyle{\subnum} El problema del área bajo la curva:\par
\numstyle{\subsubnum} Sumatorias finitas:\par
\numstyle{\subsubsubnum} Propiedades de las sumatorias finitas.\par
\numstyle{\subsubsubnum} Aproximaciones inferior, superior y media.\par
\numstyle{\subsubnum} Sumas de Riemann:\par
\numstyle{\subsubnum} Definición de la integral definida como límite de sumas de Riemann.\par

\numstyle{\subnum} La integral definida:\par
\numstyle{\subsubnum} Propiedades de la integral definida.\par
\numstyle{\subsubsubnum} En funciones pares e impares.\par
\numstyle{\subsubnum} Teorema del Valor Medio para integrales.\par

\numstyle{\subnum} La integral indefinida:\par
\numstyle{\subsubnum} Antiderivadas o primitivas de una función.\par
\numstyle{\subsubnum} Reglas básicas de integración.\par
\numstyle{\subsubnum} Integrales de funciones elementales.\par
\numstyle{\subsubnum} Integrales de funciones más específicas:\par
\numstyle{\subsubsubnum} Logaritmo natural y función exponencial.\par
\numstyle{\subsubsubnum} Funciones trigonométricas inversas e hiperbólicas.\par
\numstyle{\subnum} El Teorema Fundamental del Cálculo:\par
\numstyle{\subsubnum} Primera parte (relación entre derivadas e integrales).\par
\numstyle{\subsubnum} Segunda parte (cálculo de integrales definidas).\par
\end{seccionbox}

% ================= SECCIÓN 2 =================
\setcounter{subsec}{0}
\setcounter{subsubsec}{0}
\begin{seccionbox}{color2}{\mainnum. Técnicas de Integración}
\vspace{0.5em}
\numstyle{\subnum} Integración por $u$-sustitución (cambio de variable).\par
\numstyle{\subnum} Integración por partes:\par
\numstyle{\subsubnum} Fórmula, ejemplos y aplicaciones iterativas.\par
\numstyle{\subsubnum} Regla ILATE.\par
\numstyle{\subnum} Integrales trigonométricas:\par
\numstyle{\subsubnum} Productos de potencias de senos y cosenos.\par
\numstyle{\subsubnum} Otras identidades trigonométricas.\par
\numstyle{\subnum} Sustitución trigonométrica:\par
\numstyle{\subsubnum} Para $\sqrt{a^2 - x^2}$, $x = a\sin(\theta)$.\par
\numstyle{\subsubnum} Para $\sqrt{x^2 - a^2}$, $x = a\sec(\theta)$.\par
\numstyle{\subsubnum} Para $\sqrt{a^2 + x^2}$, $x = a\tan(\theta)$.\par
\numstyle{\subsubnum} Uso de triángulos rectángulos para generar la sustitución.\par
\numstyle{\subnum} Integración por fracciones parciales:\par
\numstyle{\subsubnum} Denominadores lineales y lineales repetidos.\par
\numstyle{\subsubnum} Denominadores cuadráticos y cuadráticos repetidos.\par
\numstyle{\subnum} Sustituciones especiales:\par
\numstyle{\subsubnum} Sustitución Universal (de Weierstrass), $\tan \left( \frac{\theta}{2} \right)$.\par
\numstyle{\subnum} Integración numérica:\par
\numstyle{\subsubnum} Regla del Trapecio y de Simpson.\par
\numstyle{\subsubnum} Estimación del error.\par
\end{seccionbox}

% ================= SECCIÓN 3 =================
\setcounter{subsec}{0}
\setcounter{subsubsec}{0}
\begin{seccionbox}{color3}{\mainnum. Aplicaciones de la Integración}
\vspace{0.5em}
\numstyle{\subnum} Área entre curvas:\par
\numstyle{\subsubnum} Integración con respecto a $x$ y $y$.\par
\numstyle{\subsubnum} Casos con corrimientos (ejes no necesariamente $x=0$ o $y=0$).\par
\numstyle{\subnum} Volúmenes de sólidos de revolución:\par
\numstyle{\subsubnum} Método de discos y arandelas.\par
\numstyle{\subsubnum} Método de cascarones cilíndricos.\par
\numstyle{\subsubnum} Alrededor de ejes arbitrarios.\par
\numstyle{\subnum} Longitud de arco de curvas planas.\par
\numstyle{\subnum} Áreas de superficies de revolución.\par
\numstyle{\subnum} Otras aplicaciones:\par
\numstyle{\subsubnum} Funciones de densidad de probabilidad y valor esperado.\par
\numstyle{\subsubnum} Momentos, centros de masa y centroides.\par
\numstyle{\subsubsubnum} Teorema de Pappus.\par
\numstyle{\subsubnum} Trabajo y fuerza de fluidos.\par

\end{seccionbox}

% ================= SECCIÓN 4 =================
\setcounter{subsec}{0}
\setcounter{subsubsec}{0}
\begin{seccionbox}{color4}{\mainnum. Integrales Impropias}
\vspace{0.5em}
\numstyle{\subnum} Integrales con límites infinitos:\par
\numstyle{\subsubnum} Evaluación mediante límites.\par
\numstyle{\subsubnum} Criterios de convergencia (comparación, límite comparado).\par
\numstyle{\subnum} Integrales con integrando no acotado:\par
\numstyle{\subsubnum} Integrales con discontinuidades.\par
\numstyle{\subsubnum} Criterios de convergencia.\par
\end{seccionbox}

% ================= SECCIÓN 5 =================
\setcounter{subsec}{0}
\setcounter{subsubsec}{0}
\begin{seccionbox}{color5}{\mainnum. Sucesiones y Series}
\vspace{0.5em}
\numstyle{\subnum} Sucesiones numéricas:\par
\numstyle{\subsubnum} Definición y propiedades.\par
\numstyle{\subsubnum} Límites de sucesiones.\par
\numstyle{\subsubnum} Convergencia y divergencia.\par
\numstyle{\subnum} Series numéricas:\par
\numstyle{\subsubnum} Definición y propiedades.\par
\numstyle{\subsubnum} Series armónicas y geométricas.\par
\numstyle{\subsubnum} Pruebas de convergencia (y sus condiciones):\par
\numstyle{\subsubsubnum} Prueba de divergencia.\par
\numstyle{\subsubsubnum} Prueba de comparación.\par
\numstyle{\subsubsubnum} Prueba del cociente.\par
\numstyle{\subsubsubnum} Prueba de la raíz.\par
\numstyle{\subsubsubnum} Prueba de la integral.\par
\numstyle{\subsubnum} Series alternantes y convergencia absoluta.\par
\numstyle{\subnum} Series de potencias:\par
\numstyle{\subsubnum} Radio e intervalo de convergencia.\par
\numstyle{\subsubnum} Diferenciación e integración término a término.\par
\numstyle{\subnum} Series de Taylor y MacLaurin:\par
\numstyle{\subsubnum} Aproximación de funciones.\par
\numstyle{\subsubnum} Series de Taylor alrededor de un punto $c$.\par
\numstyle{\subsubnum} Series de MacLaurin $(c=0)$.\par
\numstyle{\subsubnum} Aplicaciones y estimación del error.\par
\end{seccionbox}

\end{document}
