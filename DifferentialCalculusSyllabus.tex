\documentclass[a4paper, spanish]{extarticle} % Clase de documento tipo artículo extendido, papel A4, idioma español
\usepackage[spanish]{babel} % Traduce elementos automáticos al español
\usepackage[T1]{fontenc} % Codificación de fuentes para caracteres acentuados
\usepackage{geometry}  % Permite modificar los márgenes de la página
\usepackage{tcolorbox} % Cajas de color personalizables (para secciones)
\usepackage{xcolor} % Definición y uso de colores
\usepackage{sourcesanspro} % Fuente sans-serif moderna
\usepackage{lmodern} % Fuente Latin Modern
\usepackage{microtype} % Mejora la tipografía (espaciado, guiones, etc.)
\usepackage{expl3} % Sintaxis y utilidades avanzadas de LaTeX3
\usepackage{xparse} % Definición avanzada de comandos
\usepackage{amssymb} % Símbolos matemáticos adicionales

% Modifica el tamaño y espaciado de la fuente normal del documento
\makeatletter
\renewcommand\normalsize{%
   \@setfontsize\normalsize{10}{12}%  % 10pt de tamaño, 12pt de interlineado
   \abovedisplayskip 10\p@ \@plus2\p@ \@minus5\p@
   \abovedisplayshortskip \z@ \@plus3\p@
   \belowdisplayshortskip 6\p@ \@plus3\p@ \@minus3\p@
   \belowdisplayskip \abovedisplayskip
   \let\@listi\@listI}
\normalsize
\makeatother

% --------- MACROS para títulos estilizados ---------
\ExplSyntaxOn
% Crea títulos grandes y medianos para letras y números (usado en el título principal)
\NewDocumentCommand{\medstyledtitle}{m}
{
  \group_begin:
  \str_map_inline:nn {#1}
  {
    \__medstyledtitle_process:n {##1}
  }
  \group_end:
}

\cs_new_protected:Nn \__medstyledtitle_process:n
{
  \str_case_e:nnF { \tl_upper_case:n {#1} }
  {
    % Definición de formato para letras, números y símbolos
    {A}{\textbf{\fontsize{24pt}{26pt}\selectfont A}}
    % ... Resto de letras y números omitidos por brevedad ...
    { }{\hspace{0.45em}}
    {:}{\fontsize{17pt}{19pt}\selectfont :}
  }
  {\fontsize{17pt}{19pt}\selectfont #1} % Por defecto, tamaño mediano
}

\NewDocumentCommand{\smallmedstyledtitle}{m}
{
  \group_begin:
  \str_map_inline:nn {#1}
  {
    \__smallmedstyledtitle_process:n {##1}
  }
  \group_end:
}

\cs_new_protected:Nn \__smallmedstyledtitle_process:n
{
  \str_case_e:nnF { \tl_upper_case:n {#1} }
  {
    % Definición de formato para letras, números y símbolos (tamaño menor)
    {A}{\textbf{\fontsize{19pt}{21pt}\selectfont A}}
    % ... Resto de letras y números omitidos por brevedad ...
    { }{\hspace{0.45em}}
    {:}{\fontsize{17pt}{19pt}\selectfont :}
  }
  {\fontsize{17pt}{19pt}\selectfont #1}
}
\ExplSyntaxOff

% --------- Configuración tipográfica general ---------
\renewcommand{\familydefault}{\sfdefault} % Fuente sans-serif por defecto
\AtBeginDocument{%
  \SetMathAlphabet{\mathrm}{normal}{T1}{lmr}{m}{n}   % Configuración de fuentes matemáticas
  \SetMathAlphabet{\mathbf}{normal}{T1}{lmr}{b}{n}
  \SetMathAlphabet{\mathit}{normal}{T1}{lmr}{m}{it}
  \SetMathAlphabet{\mathtt}{normal}{T1}{lmtt}{m}{n}
}

\geometry{margin=2cm} % Márgenes de 2cm en toda la página

% Definición de colores personalizados para las cajas de secciones
\definecolor{color1}{HTML}{4E79A7}
\definecolor{color2}{HTML}{F28E2B}
\definecolor{color3}{HTML}{E15759}
\definecolor{color4}{HTML}{59A14F}
\definecolor{color5}{HTML}{EDC948}
\definecolor{color6}{HTML}{F28E2B}

% Configuración de la caja de sección personalizada usando tcolorbox
\tcbuselibrary{skins, breakable}
\newtcolorbox{seccionbox}[2]{
  colback=#1!5!white,              % Color de fondo suave
  colframe=#1!90!black,            % Borde oscuro del mismo color
  fonttitle=\bfseries\large,       % Fuente del título en negrita y grande
  title={\sffamily #2},            % Título en sans-serif
  breakable,
  enhanced,
  boxrule=1.5pt,                   % Grosor del borde
  arc=4pt,                         % Esquinas redondeadas
  attach boxed title to top left={xshift=5mm, yshift=-2mm},
  boxed title style={
    colback=#1!80!black,
    arc=3pt,
    outer arc=3pt
  }
}

% --------- Contadores y comandos para numeración automática de secciones ---------
\newcounter{mainsec}
\newcounter{subsec}[mainsec]
\newcounter{subsubsec}[subsec]
\newcounter{subsubsubsec}[subsubsec]

\newcommand{\mainnum}{\stepcounter{mainsec}\arabic{mainsec}}
\newcommand{\subnum}{\stepcounter{subsec}\arabic{mainsec}.\arabic{subsec}}
\newcommand{\subsubnum}{\stepcounter{subsubsec}\arabic{mainsec}.\arabic{subsec}.\arabic{subsubsec}}
\newcommand{\subsubsubnum}{\stepcounter{subsubsubsec}\arabic{mainsec}.\arabic{subsec}.\arabic{subsubsec}.\arabic{subsubsubsec}}
\newcommand{\numstyle}[1]{\textbf{\large #1}\quad} % Estilo destacado para la numeración

% --------- Título personalizado con letras grandes ---------
\title{\vspace{-2em}%
  \medstyledtitle{C}%
  \smallmedstyledtitle{ÁLCULO}%
  \hspace{0.45em}%
  \medstyledtitle{D}%
  \smallmedstyledtitle{IFERENCIAL}%
}
\author{}
\date{}

\begin{document}
\maketitle

% Reinicio de contadores para numeración limpia
\setcounter{mainsec}{0}
\setcounter{subsec}{0}
\setcounter{subsubsec}{0}
\setcounter{subsubsubsec}{0}

\begin{seccionbox}{color1}{\mainnum. Preliminares}
\vspace{0.5em}
\numstyle{\subnum} Conjuntos Numéricos y sus Propiedades: \par
\numstyle{\subsubnum} Números naturales $(\mathbb{N})$, enteros $(\mathbb{Z})$, racionales $(\mathbb{Q})$ e irracionales $(\mathbb{I})$. \par
\numstyle{\subsubnum} Números reales $(\mathbb{R})$. \par

\numstyle{\subnum} Axiomas de los Números Reales: \par
\numstyle{\subsubnum} Propiedades de cuerpo, orden y completitud. \par

\numstyle{\subnum} Desigualdades, Inecuaciones y Valor Absoluto: \par
\numstyle{\subsubnum} Intervalos: Notación y representación gráfica. \par
\numstyle{\subsubnum} Resolución de desigualdades e inecuaciones. \par
\numstyle{\subsubnum} Valor absoluto: Definición y propiedades. \par
\end{seccionbox}

% Reinicio de contadores para nueva sección principal
\setcounter{subsec}{0}
\setcounter{subsubsec}{0}
\setcounter{subsubsubsec}{0}

\begin{seccionbox}{color2}{\mainnum. Funciones y Modelos}
\vspace{0.5em}
\numstyle{\subnum} Conceptos Fundamentales de Funciones: \par
\numstyle{\subsubnum} Definición, notación y representaciones. \par
\numstyle{\subsubnum} Dominio y rango. \par
\numstyle{\subsubnum} Álgebra de funciones. \par
\numstyle{\subsubnum} Composición de funciones. \par
\numstyle{\subsubnum} Función inversa: Definición y determinación algebraica. \par
\numstyle{\subsubnum} Clasificaciones de funciones: Inyectivas, sobreyectivas y biyectivas. \par

\numstyle{\subnum} Tipos de Funciones: \par
\numstyle{\subsubnum} Funciones polinomiales y racionales: \par
\numstyle{\subsubsubnum} Operaciones básicas. \par
\numstyle{\subsubsubnum} Grado, raíces y comportamiento. \par
\numstyle{\subsubsubnum} Técnicas de Factorización. \par
\numstyle{\subsubsubnum} División polinomial (Regla de Ruffini). \par
\numstyle{\subsubsubnum} Simplificación de expresiones y racionalización. \par
\numstyle{\subsubnum} Funciones exponenciales y logarítmicas. \par
\numstyle{\subsubnum} Funciones trigonométricas e hiperbólicas: \par
\numstyle{\subsubsubnum} Funciones recíprocas e inversas. \par
\numstyle{\subsubsubnum} Identidades trigonométricas. \par
\numstyle{\subsubnum} Funciones definidas por partes: \par
\numstyle{\subsubsubnum} Valor absoluto y función mayor entero. \par

\numstyle{\subnum} Gráficas y Transformaciones: \par
\numstyle{\subsubnum} Gráficas de funciones elementales. \par
\numstyle{\subsubnum} Transformaciones gráficas: \par
\numstyle{\subsubsubnum} Traslaciones (horizontales y verticales). \par
\numstyle{\subsubsubnum} Escalamientos (compresiones y estiramientos). \par
\numstyle{\subsubsubnum} Reflexiones (respecto a los ejes $x$ e $y$). \par
\numstyle{\subsubnum} Comportamiento asintótico. \par
\end{seccionbox}

% Reinicio de contadores para nueva sección principal
\setcounter{subsec}{0}
\setcounter{subsubsec}{0}
\setcounter{subsubsubsec}{0}

\begin{seccionbox}{color3}{\mainnum. Límites y Continuidad}
\vspace{0.5em}
\numstyle{\subnum} Fundamentos de Límites: \par
\numstyle{\subsubnum} Definición intuitiva y formal $(\epsilon - \delta)$. \par
\numstyle{\subsubnum} Límites laterales. \par
\numstyle{\subsubnum} Límites que no existen. \par

\numstyle{\subnum} Cálculo de Límites: \par
\numstyle{\subsubnum} Propiedades de los límites. \par
\numstyle{\subsubnum} Sustitución directa y técnicas algebraicas para formas indeterminadas. \par
\numstyle{\subsubnum} Teorema del emparedado. \par

\numstyle{\subnum} Límites Infinitos y en el Infinito: \par
\numstyle{\subsubnum} Límites infinitos y asíntotas verticales. \par
\numstyle{\subsubnum} Límites en el infinito y asíntotas horizontales. \par

\numstyle{\subnum} Continuidad: \par
\numstyle{\subsubnum} Definición en un punto y en un intervalo. \par
\numstyle{\subsubnum} Tipos de discontinuidades: Removibles, de salto, infinitas. \par
\numstyle{\subsubnum} Teorema del Valor Intermedio. \par
\numstyle{\subsubnum} Teorema del Valor Extremo. \par
\end{seccionbox}

% Reinicio de contadores para nueva sección principal
\setcounter{subsec}{0}
\setcounter{subsubsec}{0}
\setcounter{subsubsubsec}{0}

\begin{seccionbox}{color4}{\mainnum. La Derivada}
\vspace{0.5em}
\numstyle{\subnum} Introducción a la Derivada: \par
\numstyle{\subsubnum} Definición formal de la derivada (límite). \par
\numstyle{\subsubnum} Razón de cambio promedio e instantánea. \par
\numstyle{\subsubnum} Pendiente y recta tangente. \par
\numstyle{\subsubnum} Interpretación geométrica. \par

\numstyle{\subnum} Reglas de Diferenciación: \par
\numstyle{\subsubnum} Reglas de la constante, la potencia, la suma y la diferencia. \par
\numstyle{\subsubnum} Reglas del producto y del cociente. \par
\numstyle{\subsubnum} Regla de la cadena. \par

\numstyle{\subnum} Derivadas de Funciones Específicas: \par
\numstyle{\subsubnum} Derivadas de funciones polinomiales. \par
\numstyle{\subsubnum} Derivadas de funciones exponenciales y logarítmicas. \par
\numstyle{\subsubnum} Derivadas de funciones trigonométricas, trigonométricas inversas e hiperbólicas. \par

\numstyle{\subnum} Técnicas Adicionales de Diferenciación: \par
\numstyle{\subsubnum} Diferenciación implícita y derivación logarítmica. \par
\numstyle{\subsubnum} Derivadas de orden superior. \par
\end{seccionbox}

% Reinicio de contadores para nueva sección principal
\setcounter{subsec}{0}
\setcounter{subsubsec}{0}
\setcounter{subsubsubsec}{0}

\begin{seccionbox}{color5}{\mainnum. Aplicaciones de la Diferenciación}
\vspace{0.5em}
\numstyle{\subnum} Análisis de Funciones con Derivadas: \par
\numstyle{\subsubnum} Teorema de Rolle y Teorema del Valor Medio. \par
\numstyle{\subsubnum} Valores extremos de funciones: Máximos y mínimos absolutos y relativos. \par
\numstyle{\subsubnum} Monotonía (crecimiento o decrecimiento de funciones). \par
\numstyle{\subsubnum} Concavidad y puntos de inflexión. \par
\numstyle{\subsubnum} Criterios de la primera y segunda derivada. \par

\numstyle{\subnum} Aplicaciones y Métodos: \par
\numstyle{\subsubnum} Trazado de curvas. \par
\numstyle{\subsubnum} Aproximaciones lineales y diferenciales. \par
\numstyle{\subsubnum} Tasas de cambio relacionadas. \par
\numstyle{\subsubnum} Optimización. \par
\numstyle{\subsubnum} Método de Newton-Raphson. \par
\numstyle{\subsubnum} Regla de L'Hôpital para formas indeterminadas. \par
\end{seccionbox}

\end{document}
